Como parte del programa de salud de la Universidad Autónoma de Bucaramanga, se ha decidido realizar un tamizaje de peso y talla y el correspondiente cálculo del índice de masa corporal a cada persona de la comunidad universitaria, entregando recomendaciones dependiendo del resultado arrojado por la fórmula de índice de masa corporal.

Se define el Índice de Masa Corporal -> IMC = peso [kg]/ (estatura [m])^2

El programa debe permitir al menos:

1. App de usuario: Ingreso de datos de talla, peso y contacto a cada persona de la comunidad universitaria
2. App de usuario: Al finalizar un registro de datos, el usuario recibe indicaciones estandarizadas (recomendaciones nutricionales y de ejercicio) dependiendo de su clasificación a partir del IMC
3. App de usuario: El usuario podrá realizar varios registros en diferentes fechas y ver su historial

4. App de bienestar universitario: Ingreso de personal de bienestar universitario, para ver los datos recopilados e informes acumulados sobre estos datos.